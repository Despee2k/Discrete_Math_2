\documentclass[letterpaper]{article}
\usepackage[extreme]{savetrees}
\usepackage[utf8]{inputenc}
\usepackage{amssymb}
\usepackage{amsmath}

\title {Lecture 1 Assignment}
\author {Daganta - 23102320}
\hbadness=99999

\begin {document}
\maketitle
\begin{enumerate}

    \item Recall the definition of a rational number, denoted as $\mathbb{Q}$. 
    Prove that the Euler's number $e=\sum_{k=0}^{\infty} \frac{1}{k!} \notin \mathbb{Q}$. 
    A factorial is defined as $k! = (k)(k-1)(k-2)(k-3)...,\forall k \in \mathbb{Z}^{+}$, note that $0 != 1$. 
    Furthermore, a sum notation $\sum_{k=0}^{\infty} k=0+1+2+3+...+...$\\

        Suppose that $e$ is rational, such that $e = \frac{n}{d}$, where $n$ and $d$ are positive integers.
            $$\frac{n}{d}=e=\sum_{k=0}^{\infty} \frac{1}{k!}$$
        Multiply both sides by $d!$, we get the following equation
            \begin{equation*}
                \begin{split}
                    \frac{n}{d}d! & = d!\sum_{k=0}^{\infty} \frac{1}{k!}                                                                                                                                             \\
                    n(d - 1)!     & = d! \left[\frac{1}{0!} + \frac{1}{1!} + \frac{1}{2!} + \dots + \frac{1}{d!} + \frac{1}{(d+1)!} + \frac{1}{(d+2)!} + \frac{1}{(d+3)!} + \dots \right]                        \\
                                & = \left(\frac{d!}{0!} + \frac{d!}{1!} + \frac{d!}{2!} + \dots + \frac{d!}{d!}\right) + \left[\frac{d!}{(d + 1)!}  + \frac{d!}{(d + 2)!} + \frac{d!}{(d + 3)!} + \dots\right] \\
                                & = \sum_{k=0}^{d} \frac{d!}{k!} + \sum_{k=d+1}^{\infty} \frac{d!}{k!}
                \end{split}
            \end{equation*}  
        \\
        The term $n(d - 1)!$ is clearly an integer. The first sum is also an integer, since $k \leq d$. The second sum we can simplify as follows
                $$\frac{1}{d+1} + \frac{1}{(d+1)(d+2)} + \frac{1}{(d+1)(d+2)(d+3)} + \dots$$
        \\
        The second sum is clearly greater than $0$, and we can see (by considering respective terms) that
            $$\left[\frac{1}{d+1} + \frac{1}{(d+1)(d+2)} + \frac{1}{(d+1)(d+2)(d+3)} + \dots\right] < \sum_{k=0}^{\infty} \left(\frac{1}{d+1}\right)^{(k+1)}$$
        \\
        The right-hand side of the inequality is a geometric series, and we used the formula to show that the second sum is going to be less than $1$
            \begin{equation*}
                \begin{split}
                    \sum_{k=0}^{\infty} \left(\frac{1}{d+1}\right)^{(k+1)} & = \left[\frac{1}{d+1} + \frac{1}{(d+1)^2} + \frac{1}{(d+1)^3} + \dots\right] \\
                                                                            & = \frac{\frac{1}{d+1}}{1 - \frac{1}{d+1}} = \frac{1}{d} \leq 1
                \end{split}
            \end{equation*}
        \\
        With this logical statement, the second sum is greater than $0$ and less than $1$, which is not an integer. This is a contradiction, since $n(d-1)!$ is an integer, and the second sum is not an integer
        $$n(d - 1)! \neq \sum_{k=0}^{d} \frac{d!}{k!} + \sum_{k=d+1}^{\infty} \frac{d!}{k!}$$
        \\
        Therefore, $e$ is irrational since $\mathbb{Z} = \mathbb{Z} + \overline{\mathbb{Z}}$ is simply impossible.
        \\
    
    \item Prove Minkowski's Inequality for sums, $\forall (p > 1, (a_k, b_k) > 0)$:
    \\
        $$\left(\sum_{k=1}^n |a_k + b_k|^p\right)^\frac{1}{p} \leq \left(\sum_{k=1}^n |a_k|^p\right)^\frac{1}{p} + \left(\sum_{k=1}^n |b_k|^p\right)^\frac{1}{p}$$\\
        
        Define:\\
            $$q=\frac{p}{p-1}$$\\
        Then:\\
            $$\frac{1}{p}+\frac{1}{q}=\frac{1}{p}+\frac{p-1}{p} = 1$$\\
        It follows that:\\
        \begin{equation*}
            \begin{split}
                \sum_{k=1}^{n} (a_k + b_k) & = \sum_{k=1}^{n} a_k (a_k + b_k)^{p-1} + \sum_{k=1}^{n} b_k (a_k + b_k)^{p-1} \\
                & \leq \left( \sum_{k=1}^{n} {a_k}^p \right)^{\frac{1}{p}} \left( \sum_{k=1}^{n} \left( (a_k + b_k)^{p-1} \right)^q \right)^{\frac{1}{q}} + \left( \sum_{k=1}^{n} b_k^p \right)^{\frac{1}{p}} \left( \sum_{k=1}^{n} \left( (a_k + b_k)^{p-1} \right)^q \right)^{\frac{1}{q}} \\
                & \quad \text{(Holder's Inequality for Sums, applied twice)} \\
                & = \left( \sum_{k=1}^{n} {a_k}^p \right)^{\frac{1}{p}} \left( \sum_{k=1}^{n} (a_k + b_k)^p \right)^{\frac{1}{q}} + \left( \sum_{k=1}^{n} b_k^p \right)^{\frac{1}{p}} \left( \sum_{k=1}^{n} (a_k + b_k)^p \right)^{\frac{1}{q}} \\
                & \quad \text{(by the power of power property and hypothesis: $(p-1)q = p$)} \\
                & = \left(\left(\sum_{k=1}^{n} {a_k}^p \right)^{\frac{1}{p}} + \left(\sum_{k=1}^{n} {b_k}^p \right)^{\frac{1}{p}}\right) \left(\sum_{k=1}^{n} (a_k + b_k)^p \right)^{\frac{1}{q}}\\
            \end{split}
        \end{equation*}
        \begin{flalign*}
            \left(\sum_{k=1}^{n}(a_k+b_k)^p\right)^{1-\frac{1}{q}} \leq \left(\sum_{k=1}^{n} {a_k}^p\right)^{\frac{1}{p}} + \left(\sum_{k=1}^{n} {b_k}^p\right)^\frac{1}{p} \quad \text{(Dividing by:)}\left(\sum_{k=1}^{n}(a_k+b_k)^p\right)^{\frac{1}{q}}&&
        \end{flalign*}
        \begin{flalign*}
            \left(\sum_{k=1}^{n}(a_k+b_k)^p\right)^\frac{1}{p} \leq \left(\sum_{k=1}^{n} {a_k}^p\right)^{\frac{1}{p}} + \left(\sum_{k=1}^{n} {b_k}^p\right)^\frac{1}{p} \quad \text{(as $1 - \frac{1}{q} = p$)}&&
        \end{flalign*}\\

  \item Prove the triangle inequality $|x  +y| \leq |x| + |y|, \forall(x,y) \in \mathbb{R}$.\\
  Proof of Triangle Inequality:\\
    For any real numbers $x$ and $y$, we want to prove that $|x + y| \leq |x| + |y|$.\\ \\
    Consider the cases:\\ \\
    Case 1: $x + y \geq 0$\\
    In this case, $|x + y| = x + y$, $|x| = x$, and $|y| = y$. Therefore, we have:\\
    $ |x + y| = x + y = |x| + |y| $\\ \\
    Case 2: $x + y < 0$\\
    In this case, $|x + y| = -(x + y)$, $|x| = -x$, and $|y| = -y$. Therefore, we have:\\
    $ |x + y| = -(x + y) = -x - y = |x| + |y| $\\ \\
    Conclusion:\\
    In both cases, we have $|x + y| \leq |x| + |y|$. Thus, the triangle inequality holds for all real numbers $x$ and $y$.

  \item Prove Sedrakyan's Lemma $\forall u_i, v_i \in \mathbb{R}^+$:
        \setcounter{equation}{0}
        \begin{equation}
            \frac{(\sum_{i = 1}^n u_i)^2}{\sum_{i = 1}^n v_i} \leq \sum_{i=1}^n \frac{(u_i)^2}{v_i}
        \end{equation}

        To prove Sedrakyan's Lemma, we can use Cauchy-Schwarz Inequality.

        Let's denote $x_i = \sqrt{vi}$ and $y_i = u_i$. Then the given inequality becomes

        \begin{equation}
            \left(\frac{\sum_{i=1}^{n}x_iy_i}{\sqrt{\sum_{i=1}^{n}v_i}}\right)^2 \leq \sum_{i=1}^{n}\frac{(x_i)^2 (y_i)^2}{v_i}
        \end{equation}

        This is essentially the squared form of the Cauchy-Schwarz Inequality, which states

        \begin{equation}
            \left(\sum_{i=1}^{n}a_i b_i\right)^2 \leq \left(\sum_{i=1}^{n}a_i^2\right) \left(\sum_{i=1}^{n}b_i^2\right)
        \end{equation}

        where

        \begin{align}
            a_i & = x_i\sqrt{\frac{v_i}{\sum_{i=1}^{n}v_i}}          \\
            b_i & = \frac{y_i}{\sqrt{\frac{v_i}{\sum_{i=1}^{n}v_i}}}
        \end{align}

        By applying the Cauchy-Schwarz Inequality to $(4)$ and $(5)$, we get

        \begin{equation}
            \left(\sum_{i=1}^{n}x_i y_i \sqrt{\frac{v_i}{\sum_{i=1}^{n}v_i}}\right)^2 \leq \left(\sum_{i=1}^{n}(x_i)^2 \frac{v_i}{\sum_{i=1}^{n}v_i}\right) \left(\sum_{i=1}^{n}(y_i)^2 \frac{1}{\frac{v_i}{\sum_{i=1}^{n}v_i}}\right)
        \end{equation}

        Simplify

        \begin{equation}
            \left(\sum_{i=1}^{n}x_i y_i\right)^2 \leq \left(\sum_{i=1}^{n}(x_i)^2\right) \left(\sum_{i=1}^{n}(y_i)^2\right)
        \end{equation}

        Which is equivalent to

        \begin{equation}
            \left(\frac{\sum_{i=1}^{n}u_i}{\sqrt{\sum_{i=1}^{n}v_i}}\right)^2 \leq \sum_{i=1}^{n}\frac{(u_i)^2}{v_i}
        \end{equation}

        Therefore, Sedrakyan's Lemma is proved.
\end{enumerate}
\end {document}